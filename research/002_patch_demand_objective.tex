\documentclass[11pt]{article}

\usepackage{amsmath, amssymb}
\usepackage{geometry}
\geometry{margin=1in}

\title{Research Note 002:\\
Patch Demand Criteria for Lung Localization in Chest X-ray Images}
\author{}
\date{}

\begin{document}
\maketitle

\section{Motivation}

Previous analysis showed that lung regions in chest X-ray (CXR) images form large basin-like
structures in an image energy landscape.
However, basin-based verification methods such as gravity-inspired attraction
are sensitive to patch placement and may fail when candidate patches are oversized or imperfectly localized.

This note focuses on an earlier stage of the pipeline: \emph{patch selection}.
Rather than defining a single optimized score or decision boundary,
the goal is to characterize patch \emph{demand} through a set of anatomical plausibility criteria.

\section{Problem Setting}

Given a binary core mask $M_{\text{core}}$ derived from coarse preprocessing,
connected components are extracted as candidate regions.

The objective is to rank these candidates according to their likelihood of corresponding
to lung regions, without introducing learnable parameters or optimization-based weighting.

\section{Candidate Generation from Core Mask}

Connected components from $M_{\text{core}}$ are labeled and filtered based on a minimum area threshold.
Each remaining component $C_i$ is treated as a candidate patch.

For each $C_i$, feature extraction is conducted independently,
combining global spatial context and local texture cues.

\section{Rib-Based Local Feature Extraction}

To capture rib-induced periodic patterns, we compute the positive vertical gradient:
\[
G_y = \max\left( \frac{\partial I}{\partial y}, 0 \right)
\]

This gradient is projected into a one-dimensional signal by summation along the horizontal axis:
\begin{equation}
s_i(y) = \sum_x G_y(x,y)
\end{equation}

The rib periodicity score $S_i$ is defined as the product of two components:
\begin{equation}
S_i = S_{\text{peak}}(s_i) \cdot S_{\text{auto}}(s_i)
\end{equation}

\subsection{Peak Density Component}

The first component measures the regularity of inter-rib spacing:
\begin{equation}
S_{\text{peak}}(s_i) = \frac{n}{\sigma_{\Delta} + \epsilon}
\end{equation}
where $n$ is the number of detected peaks and $\sigma_{\Delta}$ is the standard deviation
of the peak-to-peak distances.

\subsection{Autocorrelation Component}

The second component measures pattern repetition using normalized autocorrelation.
Let $\hat{s}_i(y) = s_i(y) - \bar{s}_i$.
The autocorrelation score is defined as:
\begin{equation}
S_{\text{auto}}(s_i) =
\sum_{\tau = \tau_{\min}}^{\tau_{\max}} R_i(\tau),
\quad
R_i(\tau) =
\frac{\sum_y \hat{s}_i(y)\hat{s}_i(y-\tau)}
{\sum_y \hat{s}_i(y)^2}
\end{equation}

High values of $S_i$ indicate strong rib periodicity,
whereas lung interiors typically exhibit lower scores.

\section{Global Positional Criterion}

Let $c_i$ denote the centroid of candidate $C_i$.
The spine is treated as a global anatomical anchor,
estimated as a high-intensity ridge in the image energy landscape.

Candidates closer to the spine centerline are considered more plausible,
reflecting the anatomical layout of the thoracic cavity.

This criterion is used as a relative ordering cue rather than a hard constraint.

\section{Patch Scale Criterion}

In the absence of explicit body size estimation,
candidate scale is normalized by the full image area:
\[
S_{\text{scale}}(C_i) = \frac{|C_i|}{|\Omega|}
\]

This provides a coarse prior favoring candidates that occupy a significant portion
of the image, consistent with the large spatial extent of lung regions in CXR images.

No absolute threshold is imposed; scale is used only for relative comparison.

\section{Patch Demand Interpretation}

Patch demand is not defined as a single scalar score.
Instead, each candidate is evaluated based on a combination of:

\begin{itemize}
\item Low rib periodicity ($S_i$)
\item Anatomically plausible proximity to the spine
\item Sufficient spatial extent relative to the image
\end{itemize}

Candidates that satisfy these criteria to a greater extent
are considered higher-demand patches
and are prioritized for subsequent basin-based verification.

\section{Relationship to Basin-Based Verification}

Patch demand serves as a coarse selection mechanism,
while gravity-inspired attraction and basin containment
operate as verification steps.

This separation allows imperfect or oversized candidates
to be retained during selection,
while topological consistency is enforced later.

\section{Limitations}

The proposed criteria rely on coarse normalization and heuristic ordering.
No patient-specific body size normalization is performed,
and image framing may influence the scale criterion.

These limitations are accepted at this stage
to preserve interpretability and experimental transparency.

\section{Summary}

This note proposes a parameter-free patch demand formulation
for lung localization in chest X-ray images.
By combining rib-based local texture analysis,
global positional context, and relative spatial extent,
the method prioritizes anatomically plausible lung candidates
without relying on optimized weighting or learned parameters.

\end{document}
