\documentclass[11pt]{article}

\usepackage{amsmath, amssymb}
\usepackage{geometry}
\usepackage{bm}
\geometry{margin=1in}

\title{Research Note:\\
Is There a Chasm within Me?}
\author{}
\date{}

\begin{document}
\maketitle

\section{Motivation}

Traditional patch-based lung localization methods rely primarily on local texture cues, such as rib periodicity suppression.
However, analysis of the image intensity surface reveals that lung regions exhibit a global topographic structure:
they form wide, trapped basins in an energy landscape rather than simple local minima.

Earlier gravity-inspired formulations characterize lung regions as isotropic sinks that attract surrounding gradients.
While effective for strictly interior seeds, such formulations become sensitive to patch size and boundary crossing.
This note formalizes a more general patch selection criterion based on the existence and depth of basin structures
\emph{contained within} a patch.

\section{Image and Energy Field}

Let the input chest X-ray image be defined as a scalar field:
\begin{equation}
I : \Omega \subset \mathbb{R}^2 \rightarrow \mathbb{R}
\end{equation}

We define an associated energy (potential) field:
\begin{equation}
E(x,y) = \phi(I(x,y))
\end{equation}
where the mapping $\phi(\cdot)$ is chosen such that:
\begin{itemize}
\item Lung interior corresponds to lower energy
\item Bones and body boundaries correspond to higher energy
\end{itemize}

Typical choices include Gaussian-smoothed intensity or intensity augmented by gradient magnitude.

\section{Patch Representation}

Let a patch $P_i \subset \Omega$ be defined as a connected region centered at:
\begin{equation}
p_i = (x_i, y_i)
\end{equation}

Patch-based features (e.g.\ rib periodicity measures) provide a \emph{local structural prior}, but do not determine global lung membership.

\section{Gravity-Based Basin Attraction (Previous Formulation)}

In the gravity-inspired formulation, a patch is considered lung-like if its surrounding neighborhood exhibits a net inward energy gradient.
Formally, for an outer neighborhood $\Omega_i$ surrounding $P_i$, the inward gradient projection is defined as:
\begin{equation}
g(q; p_i) = \nabla E(q) \cdot \frac{p_i - q}{\lVert p_i - q \rVert}
\end{equation}

A patch satisfies the gravity criterion if a majority of $g(q; p_i)$ values are negative, indicating isotropic attraction.

This formulation is effective when $P_i$ lies strictly within the lung interior.
However, when patches are oversized or cross basin boundaries, the inward-gradient assumption breaks down,
leading to false rejections despite the presence of lung structure within the patch.

\section{Limitations of Pure Basin Attraction}

The gravity-based criterion implicitly assumes:
\begin{itemize}
\item The patch center lies near the basin minimum
\item The patch does not cross basin boundaries
\item Surrounding gradients are representative of basin topology
\end{itemize}

In practice, bounding box predictions and core masks may be oversized due to anatomical variability,
imperfect body masking, or dataset annotation bias.
In such cases, a patch may \emph{contain} a lung basin without being located at its minimum,
causing the gravity criterion to fail.

\section{Basin Containment Hypothesis}

To relax the strict interior-seed assumption, we propose the following hypothesis:

\begin{quote}
A patch should be considered lung-like if it contains a meaningful basin structure,
regardless of whether the patch center itself lies at the basin minimum.
\end{quote}

This shifts the objective from basin \emph{attraction} to basin \emph{existence and depth}.

\section{Internal Basin Structure}

Restrict the energy field to the patch domain:
\begin{equation}
E_{P_i} = E \big|_{P_i}
\end{equation}

Define the internal basin depth of patch $P_i$ as:
\begin{equation}
D_i = \max_{x \in \partial P_i} E(x) - \min_{x \in P_i} E(x)
\end{equation}

A patch is considered to contain a basin if:
\begin{equation}
D_i \ge \delta
\end{equation}
where $\delta$ is a minimum depth threshold controlling basin significance.

Additional constraints may require that the minimum lies sufficiently far from the patch boundary.

\section{Relationship to Gravity-Based Formulation}

The basin containment criterion generalizes the gravity-based formulation:
\begin{itemize}
\item Gravity-based attraction tests whether a patch is \emph{positioned at} a basin minimum
\item Basin containment tests whether a patch \emph{contains} a basin minimum
\end{itemize}

When a patch lies strictly inside a basin, both criteria agree.
When a patch is oversized or boundary-crossing, gravity-based attraction may fail,
while basin containment remains valid.

Thus, basin containment acts as a higher-level abstraction that subsumes gravity-based attraction
under weaker and more robust assumptions.

\section{Interpretation}

Under this formulation, lung regions are characterized not by absolute intensity or local texture alone,
but by the presence of deep, trapped basin structures within the image energy landscape.
Patch-based rib suppression provides a local prior,
while basin containment encodes global topographic semantics.

\section{Summary}

This note proposes a patch selection criterion based on basin containment and depth,
extending gravity-inspired basin attraction methods to handle oversized and imperfect patch proposals.
The formulation preserves the physical intuition of energy landscapes while relaxing assumptions
that are fragile under real-world annotation and segmentation noise.

\end{document}